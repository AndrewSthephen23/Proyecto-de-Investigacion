\frontmatter
\pagenumbering{roman}
\begin{titlepage}
    \centering
    {\scshape\large Universidad Nacional de Ingeniería \par}
    \vspace{0.3cm}
    {\scshape\large Facultad de Ciencias \par}
    \vspace{0.3cm}
    {\scshape\large Carrera de Ciencias de la Computación \par}
    \vspace{0.3cm}
    \includegraphics[width=0.5\textwidth]{frontmatter/UNIlogo.png}\\[0.5cm]
    {\LARGE\bfseries El Phishing en el Perú: Tendencias y Consecuencias como Principal Modalidad de Ciberataque en el Año 2023 \par}
    \vspace{1cm}
    {\scshape\large Proyecto de Investigación para Tesis \par}
    \vfill
    
    {\large\bfseries Andrei Steven Trujillo Armas \par}
    {\large\bfseries 20212147J \par}
    \vspace{0.5cm}
    \vfill
    
    {\large\bfseries Asesor \par}
    {\large Milussja Ivette Mejia Sanchez}
    \vspace{0.5cm}
    \vfill

    {\large Lima - Perú}\\
    {\large 2024\par}
    \vspace{1cm}
\end{titlepage}


\newpage
\section*{Resumen}
Este estudio investiga las tendencias y consecuencias del phishing como la principal modalidad de ciberataque en el Perú durante el año 2023. Adoptando un enfoque mixto que integra métodos cuantitativos y cualitativos, el estudio se clasifica como exploratorio y descriptivo. La investigación explorará cómo el phishing ha proliferado como forma predominante de ciberataque en el contexto peruano, examinando tácticas utilizadas, vectores de ataque comunes y motivaciones detrás de estos actos. Además, se centrará en proporcionar una visión detallada de los impactos observados del phishing en diversos sectores y regiones del país. La metodología incluirá la recolección de datos a través de análisis de contenido de noticias, informes de incidentes de seguridad y entrevistas con expertos en ciberseguridad. Se espera que los resultados no solo mejoren la comprensión de los efectos del phishing en el entorno digital y económico del Perú, sino que también contribuyan al desarrollo de estrategias efectivas de mitigación y prevención.

\let\cleardoublepage\clearpage
\tableofcontents
\let\cleardoublepage\clearpage
