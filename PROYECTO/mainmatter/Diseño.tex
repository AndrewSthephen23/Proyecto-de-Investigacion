\chapter{Diseño de la Investigación}

\section{Tipo de investigación}
La presente investigación se enmarca en un enfoque aplicado, dirigido a entender y mitigar los impactos del phishing como forma prevalente de ciberataque en el contexto peruano durante el año 2023. Este enfoque aplicado se centra en desarrollar estrategias efectivas para la prevención y respuesta ante esta amenaza digital emergente, con el propósito de fortalecer las defensas cibernéticas a nivel institucional y comunitario.
\begin{itemize}
    \item \textbf{Según su diseño:}
    La investigación adopta un diseño predominantemente no experimental, dado que no implica la manipulación deliberada de variables ni la alteración de condiciones de vida de los individuos. En lugar de ello, se basa en el análisis de datos recopilados sobre casos reales de phishing en el Perú durante 2023, así como en estudios de caso y entrevistas con expertos en ciberseguridad. Este enfoque permite una comprensión profunda de las dinámicas y repercusiones del phishing en el contexto local.
    \item \textbf{Según su enfoque:}
    La metodología empleada es principalmente cualitativa, enfocada en la comprensión detallada de las técnicas de ataque utilizadas por los ciberdelincuentes, así como en las respuestas organizacionales y gubernamentales frente a estas amenazas. Se emplearán análisis de contenido y técnicas de codificación para identificar patrones comunes de phishing y evaluar la efectividad de las medidas preventivas implementadas. Este enfoque cualitativo se complementa con elementos cuantitativos para proporcionar un panorama completo de la incidencia y prevalencia del phishing en el país.
    \item \textbf{Según su fuente de datos:}
    La investigación se basa en una combinación de fuentes primarias y secundarias. Se recopilarán datos directamente de informes de incidentes de seguridad, registros de ataques y bases de datos de instituciones relevantes en el ámbito de la ciberseguridad peruana. Además, se realizarán entrevistas estructuradas con profesionales y funcionarios involucrados en la gestión de incidentes cibernéticos para obtener perspectivas expertas sobre las tendencias y las estrategias de defensa.
    \item \textbf{Según su alcance:}
    El alcance de la investigación es exploratorio y descriptivo, buscando proporcionar una comprensión detallada y holística de cómo el phishing ha evolucionado como una modalidad de ciberataque predominante en el Perú durante el año 2023. Este enfoque permitirá identificar no solo las técnicas específicas utilizadas por los atacantes, sino también las consecuencias operativas, financieras y reputacionales para las organizaciones y los individuos afectados.
    \item \textbf{Según su propósito:}
    El propósito de la investigación es principalmente explicativo, ya que busca no solo documentar y describir los incidentes de phishing, sino también comprender las causas subyacentes de su proliferación y evaluar críticamente las respuestas existentes. Este enfoque explicativo es fundamental para proponer recomendaciones prácticas y políticas efectivas que fortalezcan la resiliencia cibernética del Perú frente a esta amenaza en evolución.

Este esquema proporciona una estructura clara y detallada del tipo de investigación que se está llevando a cabo, centrada en el análisis y la respuesta frente al phishing como modalidad principal de ciberataque en el Perú durante el año 2023.
\end{itemize}
\section{Diseño de investigación}
Este estudio empleará un enfoque mixto que combina elementos 
cuantitativos y cualitativos para investigar las tendencias 
y consecuencias del phishing como la principal forma de 
ciberataque en el Perú en 2023. En terminos de la investigación aplicada, se 
caracteriza por ser exploratoria y descriptiva: explorará 
las tácticas y motivaciones del phishing en el contexto 
peruano, mientras describe detalladamente sus efectos en 
diversos sectores y regiones del país. Este diseño permitirá 
una comprensión amplia y profunda del fenómeno, capturando 
las perspectivas de víctimas, expertos en ciberseguridad y 
responsables de políticas públicas, con el objetivo de 
informar estrategias efectivas de mitigación.
\section{Método de investigación}
Se realizará una recopilación exhaustiva de datos a partir de fuentes primarias y secundarias. Las fuentes primarias incluirán reportes de incidentes de seguridad cibernética proporcionados por entidades gubernamentales, empresas afectadas y organizaciones de seguridad. Las fuentes secundarias comprenderán estudios y análisis previos sobre ciberseguridad en el contexto peruano.

Posteriormente se aplicarán técnicas estadísticas para analizar la frecuencia y distribución geográfica de los ataques de phishing en el Perú durante el año 2023. Esto incluirá el uso de estadísticas descriptivas para caracterizar los tipos de ataques más comunes, los sectores más afectados, y el impacto económico estimado de los incidentes.

En el aspecto cualitativo, se llevarán a cabo entrevistas semiestructuradas con expertos en seguridad cibernética, representantes de empresas y funcionarios gubernamentales. Estas entrevistas proporcionarán insights cualitativos sobre las estrategias empleadas por los perpetradores de phishing, las vulnerabilidades específicas explotadas, y las respuestas organizativas y regulatorias adoptadas en respuesta a los ataques.

Se cumplirá con todas las consideraciones éticas pertinentes, asegurando la confidencialidad de los datos sensibles y obteniendo el consentimiento informado de los participantes en las entrevistas. Se tomarán medidas para proteger la privacidad de las organizaciones y personas que proporcionen información relevante para el estudio.

Finalmente los resultados cuantitativos y cualitativos se integrarán para proporcionar una visión completa de las tendencias del phishing en el Perú en 2023 y sus consecuencias. Esto permitirá identificar patrones emergentes, evaluar la efectividad de las medidas de mitigación existentes y proponer recomendaciones para fortalecer la infraestructura de ciberseguridad del país.
\section{Población}
La población objetivo de este estudio se centra en los usuarios activos de Internet en el Perú que están expuestos a ser víctimas de phishing como modalidad principal de ciberataque durante el año 2023. Esta población incluirá individuos que utilizan servicios en línea, como correo electrónico, redes sociales, y transacciones financieras a través de internet. Se considerarán usuarios de diversas edades, géneros y niveles socioeconómicos, con el objetivo de captar una muestra representativa de la población peruana afectada por este tipo de amenazas cibernéticas.
\section{Muestra}
La muestra seleccionada para este estudio estará compuesta por usuarios activos en el Perú que han sido afectados por ataques de phishing durante el año 2023. Se utilizará una estrategia de muestreo basada en criterios específicos para asegurar la representatividad y relevancia de los datos recopilados. Los criterios de selección incluirán la identificación de víctimas de phishing a través de reportes documentados, bases de datos de instituciones financieras y organismos de ciberseguridad, así como casos documentados de pérdidas financieras. Se emplearán técnicas de muestreo aleatorio estratificado para asegurar la diversidad geográfica y demográfica de la muestra, garantizando así una representación adecuada de las distintas regiones y perfiles de usuarios afectados por este tipo de ciberataques en el contexto peruano.
\section{Campo o lugar de estudio}
El campo de estudio para esta investigación se centra en el ciberespacio, específicamente en el contexto digital del Perú. Se llevará a cabo un análisis detallado de incidentes y casos documentados de phishing, focalizándose en la recopilación y análisis de datos disponibles en plataformas digitales públicas y privadas, así como en informes de entidades gubernamentales y empresas de ciberseguridad. Este entorno se caracteriza por:
\begin{itemize}
    \item Datos Digitales: El estudio se enfoca en la recopilación y análisis de datos digitales relevantes sobre incidentes de phishing en el Perú durante el año 2023, utilizando fuentes como informes públicos de instituciones financieras y organismos de seguridad cibernética.
    \item Diversidad Geográfica: Se considera la diversidad geográfica del Perú para entender las variaciones regionales en la incidencia y respuesta ante ataques de phishing, abarcando tanto áreas urbanas como rurales.
    \item Colaboración Institucional: Se establecerá una colaboración con entidades locales y regionales para acceder a datos relevantes y asegurar la representatividad de la muestra, garantizando el cumplimiento ético y legal en el manejo de la información sensible.
\end{itemize}
\section{Técnicas e instrumentos de recolección de datos}
Para llevar a cabo la investigación sobre phishing en el Perú durante el año 2023, se emplearán técnicas y herramientas especializadas adaptadas a la dinámica del ciberespacio y los objetivos del estudio.
\subsection{Técnica de recopilación de datos:}
\begin{itemize}
    \item Monitoreo Continuo en Tiempo Real: La principal técnica será el monitoreo continuo en tiempo real de fuentes digitales abiertas y reportes de incidentes de seguridad proporcionados por entidades gubernamentales y privadas. Esto asegurará la captura inmediata de datos relevantes para el análisis de tendencias y patrones de phishing en el Perú.
\end{itemize}
\subsection{Instrumentos de recolección de datos:}
\begin{itemize}
    \item Informes de Instituciones de Seguridad Cibernética: Se recopilarán informes y análisis proporcionados por instituciones como la Policía Nacional del Perú y empresas de seguridad cibernética reconocidas, que documentan casos de phishing y sus impactos.
    \item Entrevistas Semiestructuradas: Se realizarán entrevistas con expertos en ciberseguridad y representantes de instituciones financieras y gubernamentales para obtener insights cualitativos sobre las técnicas y consecuencias del phishing en el contexto peruano.
    \item Análisis de Datos Públicos: Se utilizarán bases de datos públicas y reportes de incidentes específicos de phishing en el Perú para analizar tendencias, métodos de ataque y sectores más afectados.
\end{itemize}
\subsection{Consideraciones éticas:}
\begin{itemize}
    \item Cumplimiento Legal y Ético: Se cumplirán las regulaciones y políticas éticas vigentes en la recopilación y uso de datos, asegurando la confidencialidad y anonimización de la información sensible obtenida.
    \item Consentimiento Informado: Cuando sea necesario, se obtendrá el consentimiento informado de los participantes en entrevistas y encuestas para garantizar la participación ética y voluntaria en la investigación.
\end{itemize}
