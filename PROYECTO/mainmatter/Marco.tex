\chapter{Marco Teórico}

\section{Antecedentes}
El phishing ha emergido como una de las modalidades más prevalentes y sofisticadas de ciberataque a nivel global, y Perú no es una excepción. Este tipo de ataque, que busca engañar a las víctimas para que revelen información confidencial, ha evolucionado significativamente, aprovechando tanto las vulnerabilidades tecnológicas como las humanas. Durante 2023, el incremento en los ataques de phishing en Perú ha suscitado un interés creciente en la comunidad de seguridad informática y en las autoridades gubernamentales, dado su impacto en la seguridad financiera y personal de los ciudadanos.

\subsection{Contexto tecnológico y social}
En el contexto tecnológico, estudios previos han demostrado cómo los cibercriminales han adaptado sus tácticas para explotar las tecnologías emergentes y las plataformas digitales. La investigación de McAfee (2020) destaca cómo la sofisticación de los ataques de phishing ha aumentado, empleando técnicas como el spear-phishing y el uso de inteligencia artificial para crear mensajes más convincentes y dirigidos.

Desde una perspectiva social, el trabajo de Jakobsson y Myers (2007) proporciona una comprensión profunda de cómo las técnicas de ingeniería social son cruciales para el éxito del phishing. Estos ataques no solo explotan las fallas tecnológicas, sino también las vulnerabilidades psicológicas humanas, como la confianza y la curiosidad. En Perú, el creciente acceso a Internet y la digitalización de servicios bancarios y gubernamentales han ampliado la superficie de ataque para los ciberdelincuentes.

\subsection{Relevancia para la Investigación}
Investigaciones recientes, como las realizadas por el equipo de FireEye (2021), han explorado la relación entre el aumento de los ataques de phishing y la pandemia de COVID-19. Este estudio resalta la importancia de la adaptación cultural y lingüística en las tácticas de phishing, subrayando la necesidad de enfoques específicos para diferentes regiones. En el contexto peruano, la investigación de Kaspersky (2022) destaca un aumento significativo en los ataques de phishing dirigidos a usuarios de servicios financieros y gubernamentales, lo que pone de relieve la necesidad de estudios más profundos en esta área.

\section{Diseño teórico}
El presente estudio se basa en un conjunto integral de bases teóricas que abordan aspectos clave relacionados con el phishing como modalidad de ciberataque. Estas bases teóricas proporcionan una estructura conceptual sólida y permiten una comprensión profunda del problema planteado.
\subsection{Teoría del phishing}
La teoría del phishing se fundamenta en la comprensión de las tácticas y técnicas utilizadas por los cibercriminales para engañar a sus víctimas. Esta teoría se ha desarrollado significativamente en el campo de la ciberseguridad, especialmente en el análisis de los métodos de ataque y las contramedidas efectivas. Sus principios fundamentales son los siguientes:
\begin{itemize}
    \item Ingeniería social: El phishing se basa en la manipulación psicológica para obtener información confidencial. La teoría explora cómo los atacantes utilizan técnicas de persuasión y engaño para ganarse la confianza de las víctimas.
    \item Evolución de las técnicas: La teoría aborda cómo las tácticas de phishing han evolucionado, desde correos electrónicos masivos hasta ataques más dirigidos y personalizados, conocidos como spear-phishing.
    \item Impacto en la seguridad: La teoría examina las consecuencias de los ataques de phishing en la seguridad personal y organizacional, incluyendo la pérdida de datos sensibles y el daño a la reputación.
    \item Contramedidas y prevención: La teoría también abarca las estrategias de defensa, como la educación y concienciación de los usuarios, así como el uso de tecnologías de detección y prevención.
\end{itemize}

\subsection{Teoría de la Ingeniería Social}
La ingeniería social es una disciplina que combina conocimientos de psicología y ciberseguridad para comprender cómo los atacantes explotan las debilidades humanas. Sus principios fundamentales incluyen:
\begin{itemize}
    \item Manipulación y persuasión: Los atacantes utilizan técnicas psicológicas para influir en el comportamiento de las víctimas, como el uso de la urgencia y la autoridad.
    \item Tácticas de engaño: La teoría explora cómo los atacantes crean escenarios convincentes para obtener información confidencial, desde correos electrónicos falsos hasta llamadas telefónicas fraudulentas.
    \item Defensas humanas: La teoría también se centra en cómo las personas pueden ser entrenadas para reconocer y resistir las tácticas de ingeniería social, destacando la importancia de la educación y la concienciación en ciberseguridad.
\end{itemize}

\subsection{Teoría de ciberseguridad}
La teoría de ciberseguridad se enfoca en la protección de sistemas de información contra amenazas y ataques. Sus principios fundamentales incluyen:
\begin{itemize}
    \item Identificación y gestión de riesgos: La teoría aborda cómo identificar y evaluar las amenazas, como el phishing, y cómo implementar medidas de mitigación adecuadas.
    \item Tecnologías de defensa: La teoría examina las diversas tecnologías utilizadas para prevenir y detectar ataques de phishing, como los filtros de correo electrónico y las soluciones de autenticación multifactor.
    \item Estrategias de respuesta: La teoría también incluye el desarrollo de estrategias para responder a incidentes de seguridad, minimizando el impacto y recuperando la integridad del sistema.
\end{itemize}
\section{Definición de términos}
A continuación, se presentan las definiciones de los términos clave utilizados en el marco teórico y la investigación, con el objetivo de proporcionar una comprensión clara de los conceptos fundamentales involucrados en el estudio del phishing como principal modalidad de ciberataque en Perú en 2023:
\begin{itemize}
    \item Phishing: Proceso de engañar a las víctimas para que revelen información confidencial, como contraseñas y números de tarjetas de crédito, a través de mensajes fraudulentos que parecen legítimos.
    \item Ingeniería social: Técnica utilizada por los atacantes para manipular psicológicamente a las víctimas y obtener información o acceso a sistemas.
    \item Spear-phishing: Tipo de phishing dirigido a individuos específicos, utilizando información personalizada para aumentar la efectividad del ataque.
    \item Ciberseguridad: Conjunto de prácticas y tecnologías diseñadas para proteger sistemas de información y datos contra ataques y accesos no autorizados.
    \item Autenticación multifactor: Método de seguridad que requiere dos o más formas de verificación para acceder a un sistema, aumentando la protección contra ataques de phishing.
    \item Educación y concienciación: Estrategias utilizadas para informar y entrenar a los usuarios sobre las amenazas de seguridad y cómo protegerse contra ellas.
    \item Tecnologías de detección y prevención: Herramientas y soluciones tecnológicas utilizadas para identificar y bloquear intentos de phishing antes de que lleguen a las víctimas.
\end{itemize}