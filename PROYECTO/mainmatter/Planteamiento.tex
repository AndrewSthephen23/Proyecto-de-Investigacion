
\chapter{Planteamiento del problema}

\section{Descripción del problema}

En la era digital actual, el phishing se ha consolidado como una de las 
modalidades de ciberataque más prevalentes y peligrosas, afectando tanto 
a individuos como a organizaciones en todo el mundo. En el contexto peruano,
durante el año 2023, se ha observado un incremento significativo en la 
frecuencia y sofisticación de estos ataques, lo que pone en evidencia 
la necesidad urgente de estudiar sus tendencias y consecuencias.

El phishing, que consiste en la obtención fraudulenta de información 
confidencial mediante la suplantación de identidad, presenta un desafío 
significativo para la ciberseguridad. Los atacantes utilizan técnicas 
cada vez más avanzadas para engañar a los usuarios y obtener acceso a 
datos sensibles, como credenciales bancarias y personales. Este 
fenómeno no solo impacta económicamente a las víctimas, sino que 
también socava la confianza en los sistemas digitales y las transacciones 
en línea.

El problema radica en la rápida evolución de las tácticas de phishing y 
la falta de conocimiento adecuado entre los usuarios sobre cómo identificar 
y protegerse contra estos ataques. A pesar de los esfuerzos en educación 
y prevención, muchas personas y empresas siguen siendo vulnerables debido 
a la sofisticación de los métodos empleados por los cibercriminales. Esto 
subraya la importancia de realizar una investigación detallada que explore 
las tendencias actuales del phishing en el Perú, identificando las 
principales estrategias utilizadas y evaluando sus consecuencias a nivel 
social y económico.

Esta investigación se propone analizar las características específicas 
de los ataques de phishing en el Perú durante el año 2023, identificando 
patrones y métodos predominantes. Además, busca evaluar el impacto 
de estos ataques en la economía y la seguridad digital del país, 
proporcionando una base sólida para el desarrollo de estrategias de 
mitigación más efectivas. En última instancia, se aspira a contribuir al 
fortalecimiento de la ciberseguridad en el Perú, promoviendo una mayor 
conciencia y preparación frente a esta amenaza creciente.

\newpage
\section{Formulación del problema}

\noindent Para abordar la problemática del phishing en el Perú en el año 2023 de manera efectiva, se plantean las siguientes preguntas:\par
\subsection{Pregunta principal}
\noindent ¿Cuáles fueron las tendencias más destacadas del phishing como modalidad de ciberataque en el Perú durante el año 2023 y cuáles fueron sus impactos más significativos? \par
\subsection{Preguntas específicas}
\noindent ¿Qué métodos y técnicas de phishing fueron más prevalentes y efectivos en el contexto peruano durante el año 2023? \par
\noindent ¿Cuáles son los principales factores que contribuyen a la vulnerabilidad de los usuarios peruanos ante los ataques de phishing? \par
\noindent ¿Qué medidas y estrategias de prevención y mitigación han resultado más eficaces para combatir el phishing en el contexto peruano? \par

\section{Objetivos de estudio}

\noindent Para delimitar nuestro trabajo de investigación y las expectativas que tenemos del mismo, se proponen los siguientes objetivos: \par
\subsection{Objetivo general}
\noindent Analizar las tendencias y consecuencias del phishing como principal modalidad de ciberataque en el Perú durante el año 2023, con el fin de identificar estrategias efectivas de prevención y mitigación. \par
\subsection{Objetivos específicos}
\begin{itemize}
    \item Identificar los métodos y técnicas de phishing más prevalentes y efectivos utilizados en el Perú durante el año 2023.
    \item Evaluar los principales factores que contribuyen a la vulnerabilidad de los usuarios peruanos ante los ataques de phishing.
    \item Proponer mejoras en la concienciación y la educación de los usuarios peruanos para reducir la incidencia del phishing en el futuro.
\end{itemize}
\section{Justificación}

La presente investigación surge por el contexto de que en la era digital actual, el phishing se ha convertido en una amenaza significativa para la seguridad de la información, afectando tanto a individuos como a organizaciones. Este tipo de ciberataque no solo implica pérdidas económicas, sino también daños a la reputación y la confianza en los sistemas digitales.

La importancia de esta investigación radica en la necesidad de entender cómo evolucionan las técnicas de phishing y cuál es su impacto específico en el contexto peruano. Dada la creciente sofisticación de estos ataques, es crucial identificar los métodos más utilizados y las razones por las cuales los usuarios siguen siendo vulnerables. Este conocimiento permitirá diseñar estrategias de prevención y mitigación más efectivas, adaptadas a las particularidades del entorno digital en Perú.

Además, la investigación tiene relevancia práctica, ya que proporcionará información valiosa para la elaboración de políticas públicas y programas de concienciación que puedan reducir la incidencia del phishing. Al analizar los impactos económicos y sociales del phishing, se podrán cuantificar las pérdidas y entender mejor cómo este tipo de ciberataque afecta a la economía y la seguridad nacional.

Este estudio contribuirá también al campo académico al ofrecer un análisis detallado de un problema contemporáneo y de rápida evolución. Los hallazgos podrán servir de base para futuras investigaciones y el desarrollo de tecnologías más avanzadas en ciberseguridad. En última instancia, mejorar la comprensión y la respuesta al phishing tendrá implicaciones positivas para la seguridad digital, la confianza en las transacciones en línea y la protección de datos personales en el Perú.


